\textbf{Equipo activo}

La infraestructura activa comprende los equipos electrónicos encargados de la transmisión, procesamiento y encaminamiento de la información dentro de una red. Estos dispositivos incluyen routers, switches, antenas, servidores de control y gestión, entre otros (Martínez, 2020).

Los equipos activos se caracterizan por poseer componentes electrónicos propios, como procesadores y memoria RAM, que les permiten realizar funciones de gestión y procesamiento de datos. Esto optimiza el flujo de información en la red, mejora la velocidad de transmisión y simplifica su administración (NewServerLife, 2023).

En el presente proyecto, los equipos activos se clasifican en las siguientes categorías: conmutación (switching), enrutamiento (routing), seguridad, comunicación inalámbrica, gestión y supervisión. La Tabla \ref{tab:equipos_activos} presenta los principales dispositivos utilizados.


\begin{table}[h!]
\centering
\caption{\textit{Equipos activos utilizados en la red del proyecto}}
\begin{tabular}{ll}
\toprule
\textbf{Categoría de equipo activo} & \textbf{Equipos} \\ \midrule
Dispositivos de conmutación & Switches, bridges \\
Dispositivos de encaminamiento & Routers, gateways \\
Dispositivos de acceso inalámbrico & Access points (AP) \\
Dispositivos de seguridad & Firewalls, IDS/IPS \\
Dispositivos de gestión & Servidores, controladores de red \\
Dispositivos de comunicación de voz & Teléfonos VoIP \\ \bottomrule
\end{tabular}
\label{tab:equipos_activos}
\end{table}
