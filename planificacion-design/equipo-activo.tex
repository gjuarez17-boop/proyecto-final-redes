\subsubsection{Equipo activo}

La infraestructura activa comprende los equipos electrónicos encargados de la transmisión, procesamiento y encaminamiento de la información dentro de la red. Estos dispositivos incluyen routers, switches, antenas y servidores de control y gestión \parencite{martinez2020infraestructura}.

Los equipos activos poseen componentes electrónicos propios, como procesadores y memoria, que permiten gestionar el tráfico, optimizar la velocidad de transmisión y simplificar la administración \parencite{newserverlife_network_equipment_concept_and_types_2023}. En este proyecto se agrupan en categorías de conmutación, enrutamiento, seguridad, comunicación inalámbrica y gestión. La Tabla \ref{tab:equipos_activos} muestra los principales dispositivos contemplados.

\begin{table}[H]
	\caption{\textit{Equipos activos utilizados en la red del proyecto}}
	\label{tab:equipos_activos}
	\centering
	\begin{tabularx}{0.8\textwidth}{lX}
			\toprule
			\textbf{Categoría de equipo activo} & \textbf{Equipos} \\
		\midrule
		Dispositivos de conmutación & Switches de acceso y núcleo. \\
		Dispositivos de encaminamiento & Routers para la salida perimetral. \\
		Dispositivos de acceso inalámbrico & Puntos de acceso con soporte PoE. \\
		Dispositivos de seguridad & Firewalls perimetrales y locales. \\
		Dispositivos de comunicación de voz & Teléfonos VoIP administrados. \\
		\bottomrule
	\end{tabularx}
\end{table}
