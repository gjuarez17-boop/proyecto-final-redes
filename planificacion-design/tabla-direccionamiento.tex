\subsection{Tabla de direccionamiento IP}

\begin{table}[H]
	\caption{\textit{Tabla de direccionamiento IP (VLSM)}}
	\label{tab:direccionamiento_ip}
	\centering
	\begin{tabularx}{\textwidth}{l l l l l l X}
			oprule
		VLAN & Nombre & Asignación & Subred & Prefijo & Gateway & Rango utilizable \\
		\midrule
		20 & ALUMNOS & DHCP & 148.220.0.0 & /22 & 148.220.0.1 & Pool: 148.220.0.2 -- 148.220.3.254 \\
		30 & PROFESORES & DHCP & 148.220.4.0 & /25 & 148.220.4.1 & Pool: 148.220.4.2 -- 148.220.4.126 \\
		40 & VOIP & Estática & 148.220.4.128 & /25 & 148.220.4.129 & Rango: 148.220.4.130 -- 148.220.4.254 \\
		50 & CÁMARAS & Estática & 148.220.5.0 & /25 & 148.220.5.1 & Rango: 148.220.5.2 -- 148.220.5.126 \\
		10 & ADMIN & Estática & 148.220.5.128 & /26 & 148.220.5.129 & Rango: 148.220.5.130 -- 148.220.5.190 \\
		\bottomrule
	\end{tabularx}
	\begin{flushleft}
			\textit{Nota.} El plan utiliza VLSM para optimizar el bloque de red 148.220.0.0/21. La puerta de enlace (gateway) es la primera IP utilizable de cada subred. Las VLAN de infraestructura (ADMIN, VOIP, CÁMARAS) emplean asignación estática; las VLAN de usuarios (ALUMNOS, PROFESORES) se asignan mediante DHCP.
	\end{flushleft}
\end{table}