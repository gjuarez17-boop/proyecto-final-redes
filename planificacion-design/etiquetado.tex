\textbf{Identificación y etiquetado de equipos y cableado}

Conforme al estándar ANSI/TIA-606-B, se implementará un sistema de identificación para todos los elementos de infraestructura de telecomunicaciones del edificio, incluyendo dispositivos activos y componentes de cableado, con el objetivo de asegurar trazabilidad, mantenimiento eficiente y gestión del ciclo de vida de la red (Fries, 2011).

\textit{Convención de identificadores:}  
Cada componente de la red será etiquetado con un identificador que indique su ubicación, tipo de equipo y rol dentro del sistema. La estructura general es:

\[
\text{[Edificio]-[Piso][Sala]-[RackID]-[U] : [Puerto/Conector]}
\]

\noindent Esta convención permite un registro ordenado de los equipos y facilita la gestión de la infraestructura.

\textit{Etiquetado de dispositivos activos:}  
Se aplicará a routers, switches, access points, cámaras CCTV y teléfonos VoIP, garantizando que cada equipo cuente con un identificador único impreso.

\textit{Etiquetado de cableado e infraestructura pasiva:}  
Se etiquetarán todos los enlaces de cableado horizontal y backbone, así como los patch panels y racks, siguiendo la misma lógica de ubicación y numeración. Esto asegura coherencia entre el registro documental y la disposición física de los componentes (Fries, 2011).


