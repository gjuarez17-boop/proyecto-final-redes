\subsection{Identificación y etiquetado}

Conforme al estándar ANSI/TIA-606-B, se implementará un sistema de identificación para todos los elementos de infraestructura de telecomunicaciones del edificio, incluyendo dispositivos activos y componentes de cableado, con el objetivo de asegurar trazabilidad, mantenimiento eficiente y gestión del ciclo de vida de la red (Fries, 2011).

\paragraph*{Convención de identificadores.}
Cada componente de la red se etiqueta con un identificador que indica su ubicación, tipo de equipo y rol dentro del sistema. La estructura general es:

\[
\text{[Edificio]-[Piso][Sala]-[RackID]-[U] : [Puerto/Conector]}
\]

\noindent Esta convención permite un registro ordenado de los equipos y facilita la gestión de la infraestructura.

\paragraph*{Etiquetado de dispositivos activos.}
La convención se aplica a routers, switches, puntos de acceso, cámaras CCTV y teléfonos VoIP, garantizando que cada equipo cuente con un identificador único impreso.

\paragraph*{Etiquetado de cableado e infraestructura pasiva.}
Todos los enlaces de cableado horizontal y del backbone, así como los patch panels y racks, se identifican con la misma lógica para mantener coherencia entre el registro documental y la disposición física de los componentes (Fries, 2011).


