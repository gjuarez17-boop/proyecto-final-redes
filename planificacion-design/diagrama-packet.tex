% Buscar imágenes en la carpeta ../images (carpeta hermana a planificacion-design)
\graphicspath{{../images/}}
    
    \textbf{Diagrama de la Red Física}
	
	La \textbf{Figura 1} muestra la topología física de la red propuesta para el edificio administrativo de tres niveles. En él se representan los equipos de cómputo, switches, routers, puntos de acceso y servidores distribuidos por planta, así como el cableado estructurado correspondiente. Este diseño busca garantizar la conectividad mediante una arquitectura jerárquica eficiente y escalable.
	
	La estructura detallada de cada rack se encuentra especificada en las \textbf{Figuras [X]} y \textbf{[Y]} de los diagramas de rack.
	
	\begin{figure}[h!]
		\centering
		\begin{subfigure}[b]{0.3\textwidth}
			\centering
			\includegraphics[width=\linewidth]{cisco-diagram-piso1}
			\caption{Piso 1}
			\label{fig:piso1}
		\end{subfigure}
		\hfill 
		\begin{subfigure}[b]{0.3\textwidth}
			\centering
			\includegraphics[width=\linewidth]{cisco-diagram-piso2}
			\caption{Piso 2}
			\label{fig:piso2}
		\end{subfigure}
		\hfill 
		\begin{subfigure}[b]{0.3\textwidth}
			\centering
			\includegraphics[width=\linewidth]{cisco-diagram-piso3}
			\caption{Piso 3}
			\label{fig:piso3}
		\end{subfigure}
		

		\caption{\textit{Topología Física de la Red del Edificio Administrativo}}
		\label{fig:red-fisica}
		
		\parbox{\textwidth}{
			\vspace{2mm}
			\small
			\textit{Nota.} Los paneles (a), (b), y (c) muestran la distribución de equipos por piso. Cada piso cuenta con su propio switch de acceso conectado a un switch de distribución central ubicado en el piso 1 (Este cumple con la función tanto core como acceso al piso 1). Esta estructura permite facilitar el mantenimiento de los equipos. Planeado para una escalabilidad a 20 años aproximadamente.
		}
	\end{figure}

	\textbf {Diagrama de conexión con el edificio de innovación}
	
	La \textbf{Figura 2} ilustra la interconexión física entre el edificio A y el edificio de innovación. Acorde a los requerimientos del proyecto. Se implementa la solución (instalación aérea) detallada en la figura.
	
	\begin{figure}[h!]
		\centering
		% Se puede ajustar el ancho si se desea, 0.6 o 0.7 es buena opción
		\includegraphics[width=0.3\textwidth]{cisco-diagram-conexion-innovacion}
		
		\caption{\textit{Topología de Interconexión Aérea con Edificio de Innovación}}
		\label{fig:conexion-innovacion} % Label único para esta figura
		
		\parbox{\textwidth}{
			\vspace{2mm}
			\small
			\textit{Nota.} La interconexión se realiza mediante un enlace de \textbf{fibra óptica aérea} multimodo. Se utiliza esta tecnología para proveer un alto ancho de banda. La conexión se establece de Router-a-Router (Capa 3) para aislar dominios de broadcast y mejorar la gestión de la red.
		}
	\end{figure}