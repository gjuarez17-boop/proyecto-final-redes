\subsubsection{Diagrama representativo}

La Figura \ref{fig:red-fisica} muestra la topología física propuesta para el edificio administrativo de tres niveles. Se representan los equipos de cómputo, switches, routers, puntos de acceso y servidores distribuidos por planta, así como el cableado estructurado asociado. El diseño prioriza una arquitectura jerárquica escalable y de fácil mantenimiento.

La estructura detallada de cada rack se complementa con los diagramas específicos incluidos en la sección de diagramas de rack.

\begin{figure}[H]
  \centering
  \begin{subfigure}[b]{0.3\textwidth}
    \centering
    \includegraphics[width=\linewidth]{cisco-diagram-piso1}
    \caption{Piso 1}
    \label{fig:piso1}
  \end{subfigure}
  \hfill
  \begin{subfigure}[b]{0.3\textwidth}
    \centering
    \includegraphics[width=\linewidth]{cisco-diagram-piso2}
    \caption{Piso 2}
    \label{fig:piso2}
  \end{subfigure}
  \hfill
  \begin{subfigure}[b]{0.3\textwidth}
    \centering
    \includegraphics[width=\linewidth]{cisco-diagram-piso3}
    \caption{Piso 3}
    \label{fig:piso3}
  \end{subfigure}
  \caption{\textit{Topología física de la red del edificio administrativo}}
  \label{fig:red-fisica}
  \begin{flushleft}
    	\textit{Nota.} Las subfiguras ilustran la distribución de equipos por piso. Cada nivel dispone de un switch de acceso enlazado al switch multicapa del primer piso, lo que facilita la escalabilidad del diseño proyectado a 20 años.
  \end{flushleft}
\end{figure}

La Figura \ref{fig:conexion-innovacion} ilustra la interconexión física entre el edificio A y el edificio de Innovación, acorde con los requerimientos establecidos.

\begin{figure}[H]
  \centering
  \includegraphics[width=0.4\textwidth]{cisco-diagram-conexion-innovacion}
  \caption{\textit{Topología de interconexión aérea con el edificio de Innovación}}
  \label{fig:conexion-innovacion}
  \begin{flushleft}
    	\textit{Nota.} La interconexión utiliza fibra óptica aérea multimodo. El enlace router a router en capa 3 aisla dominios de broadcast y optimiza la gestión de la red.
  \end{flushleft}
\end{figure}