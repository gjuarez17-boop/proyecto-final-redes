\subsection{Tamaño de red}

El proyecto contempla la implementación de una red de área local (LAN) cableada y segmentada mediante VLAN dinámicas (VLSM) para separar los servicios de telefonía IP, videovigilancia, alumnos, docentes y administración.

El backbone de fibra óptica multimodo OM3 interconecta los armarios de telecomunicaciones entre pisos y proporciona el enlace hacia el edificio de Innovación, extendiendo la infraestructura a un entorno de campus. El dimensionamiento considera nodos cableados, cámaras IP, teléfonos VoIP, equipos de red y puntos de acceso inalámbrico, con capacidad de crecimiento proyectada a 20 años.

Los elementos principales de la red se resumen en los siguientes puntos:

\begin{itemize}
  \item Tipo de red: LAN cableada (Cat 6 UTP) con integración de WLAN para usuarios móviles.
  \item Topología lógica: estrella jerárquica con switches de acceso y un switch central.
  \item Enlace interedificio: fibra óptica multimodo OM3 hacia el edificio de Innovación.
  \item Segmentación lógica: VLAN por servicio con soporte de calidad de servicio para voz.
\end{itemize}

El bloque base 148.220.0.0/21 permite 2046 hosts útiles por VLAN. Para optimizar su uso, las VLAN se subdividen con VLSM según la densidad de dispositivos de cada servicio y los requisitos de crecimiento.

\begin{table}[H]
  \caption{\textit{Resumen general de la red implementada}}
  \label{tab:resumen-red}
  \centering
  \begin{tabularx}{\textwidth}{l l l l X}
    	\toprule
    	\textbf{Tipo de red} & \textbf{Medio físico} & \textbf{Velocidad} & \textbf{Backbone} & \textbf{Segmentación} \\
    \midrule
    LAN cableada & Cobre Cat 6 UTP y fibra OM3 & 1 Gbps (cobre); 10 Gbps (fibra) & Fibra multimodo entre edificios & VLAN por servicio\\
    \bottomrule
  \end{tabularx}
\end{table}