	\textbf{Tamaño de la red (Tipo de red)}
	\par \vspace{3mm} % Espacio después del título
	
	El presente proyecto contempla la implementación de una \textbf{Red de Área Local (LAN)} cableada y segmentada mediante \textbf{VLANs} dinámicas, con el objetivo de separar los distintos servicios que conforman la infraestructura de comunicaciones: telefonía IP, videovigilancia (CCTV), red de alumnos, docentes y administración. 
	
	Adicionalmente, la red incluye un \textbf{backbone de fibra óptica multimodo OM3}, utilizado para interconectar los armarios de telecomunicaciones entre pisos y establecer el enlace principal hacia el \textbf{Edificio de Innovación}, extendiendo así la infraestructura hacia un entorno de tipo \textbf{red de campus}.
	
	La red se dimensiona considerando los nodos cableados en cada cuarto, los dispositivos de seguridad (cámaras IP), los teléfonos VoIP, los equipos de red (switches, router, etc) y los puntos de acceso inalámbrico (AP). Se prevé un \textbf{crecimiento futuro a 20 años promedio}, por lo que las subredes fueron definidas con espacio de reserva suficiente para expansión.
	
	\par \vspace{5mm} % Espacio entre secciones
	\textbf{Estructura  general de la red}
	\par \vspace{3mm}
	
	\begin{itemize}
		\item \textbf{Tipo de red:} LAN cableada (Cat6 UTP) con backbone de fibra óptica multimodo OM3.
		\item \textbf{Topología lógica:} Estrella jerárquica con switches de acceso y un switch central en el rack principal.
		\item \textbf{Tipo de enlace interedificio:} Fibra óptica multimodo (OM3), conectando el edificio A con el Edificio de Innovación.
		\item \textbf{Segmentación lógica:} Por VLANs para aislar tráfico y priorizar servicios (QoS en VoIP).
	\end{itemize}
	
	\par \vspace{5mm} % Espacio entre secciones
	\textbf{Dimensionamiento de subredes y VLANs}
	\par \vspace{3mm}
	
	El direccionamiento base definido para el proyecto es \textbf{148.x.x.x/21}, el cual permite un máximo de 2046 hosts útiles por VLAN. Sin embargo, con el fin de optimizar la administración y reducir el tamaño de los dominios de broadcast, las VLANs se subdividen en bloques menores según la cantidad de dispositivos de cada servicio.
	
	\par \vspace{5mm} % Espacio entre secciones
	\textbf{Resumen general de la red}
	
	\begin{table}[h!]
		\centering
	
		\begin{tabularx}{\textwidth}{LLLLL} 
			\toprule
			\textbf{Tipo de red} & \textbf{Medio físico} & \textbf{Velocidad} & \textbf{Backbone} & \textbf{Segmentación} \\
			\midrule
			LAN cableada y WLAN integrada & Cobre (Cat6 UTP) y fibra óptica OM3 & 1 Gbps (cobre), 10 Gbps (fibra) & Multimodo entre edificios & VLANs por servicio \\
			\bottomrule
		\end{tabularx}
		\caption{\textit{Resumen general de la red implementada}}
		\label{tab:resumen-red} % Label para la tabla
	\end{table}