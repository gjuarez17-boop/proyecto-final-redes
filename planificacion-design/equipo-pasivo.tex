\textbf{Equipo pasivo}

La infraestructura pasiva comprende todos los componentes físicos que permiten el soporte y la conexión de los equipos activos dentro de una red. Estos elementos no procesan información, pero son esenciales para la transmisión eficiente y estable de los datos (Cisneros, 2021).

Entre los principales componentes se encuentran el cableado estructurado, los conectores, los paneles de parcheo y la infraestructura de soporte físico. El cableado horizontal estará conformado por cable UTP categoría 6, encargado de establecer las conexiones entre los nodos de red y los armarios de comunicaciones. Para la interconexión principal (\textit{backbone}), se empleará fibra óptica multimodo tipo OM3, adecuada para enlaces de alta velocidad (SYSTIMAX, 2020).

La organización del cableado se realizará mediante \textit{patch panels}, racks de comunicaciones y accesorios de gestión como canaletas, bandejas y bridas, que permiten mantener el orden y facilitar el mantenimiento del sistema (Fluke Networks, 2022).

\begin{table}[h!]
\centering
\caption{\textit{Componentes pasivos utilizados en la red del proyecto}}
\begin{tabular}{ll}
\toprule
\textbf{Categoría de equipo pasivo} & \textbf{Componentes} \\ \midrule
Cableado de cobre & Cables UTP Cat6, conectores RJ-45 \\
Cableado de fibra óptica & Fibra multimodo OM3\\
Gestión de cableado & Canaletas, bandejas\\
Terminación de red & \textit{Patch panels}, \textit{faceplates} \\
Infraestructura física & Racks y gabinetes \\ \bottomrule
\end{tabular}
\label{tab:equipos_pasivos}
\end{table}

