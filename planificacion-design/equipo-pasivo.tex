\subsubsection{Equipo pasivo}

La infraestructura pasiva reúne los componentes físicos que soportan y conectan a los equipos activos. Aunque no procesan datos, son esenciales para transmitir la información de forma estable y confiable (Cisneros, 2021).

Los elementos principales incluyen el cableado estructurado, conectores, paneles de parcheo y la infraestructura de soporte. El cableado horizontal se compone de cable UTP categoría 6, encargado de enlazar los nodos con los armarios de comunicaciones, mientras que el backbone recurre a fibra óptica multimodo OM3 para enlaces de alta capacidad (SYSTIMAX, 2020). La organización se completa con \textit{patch panels}, racks y accesorios para mantener el orden del tendido (Fluke Networks, 2022).

\begin{table}[H]
	\caption{\textit{Componentes pasivos utilizados en la red del proyecto}}
	\label{tab:equipos_pasivos}
	\centering
	\begin{tabularx}{0.95\textwidth}{lX}
			oprule
			extbf{Categoría de equipo pasivo} & \textbf{Componentes} \\
		\midrule
		Cableado de cobre & Cableado UTP Cat 6 y conectores RJ-45 de alto desempeño. \\
		Cableado de fibra óptica & Fibra multimodo OM3 con terminaciones reforzadas. \\
		Gestión de cableado & Canaletas, bandejas y organizadores modulares. \\
		Terminación de red & \textit{Patch panels} y \textit{faceplates} identificados. \\
		Infraestructura física & Racks autoportantes y gabinetes de telecomunicaciones. \\
		\bottomrule
	\end{tabularx}
\end{table}