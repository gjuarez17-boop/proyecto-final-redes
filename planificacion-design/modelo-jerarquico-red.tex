
% Buscar imágenes en la carpeta ../images (carpeta hermana a planificacion-design)
\graphicspath{{../images/}}
		\textbf{Modelo de Red Jerárquico}
	\par \vspace{3mm}
	
	El diseño de red para este proyecto se fundamenta en el **modelo jerárquico de tres capas**, una arquitectura estándar de la industria promovida por empresas como Cisco. Este modelo se valora por promover la escalabilidad, el rendimiento, la facilidad de gestión y la localización de fallos (Odom, 2019).
	
	\begin{center}
		\includegraphics[width=1\linewidth]{diagrama-red-jerarquico}
	\end{center}
	
	
	El modelo jerárquico divide la red lógicamente en tres capas funcionales:
	
	\begin{itemize}
		\item \textbf{Capa de Acceso (Morado):} Es el punto de entrada a la red para los dispositivos finales (PCs, teléfonos VoIP, cámaras CCTV, APs). Su función es proveer conectividad y aplicar políticas de seguridad a nivel de puerto (Kurose \& Ross, 2021). En este proyecto, los switches de los pisos 2 y 3, así como los puertos de usuario del switch del piso 1, operan en esta capa.
		
		\item \textbf{Capa de Distribución (Azul):} Agrega el tráfico proveniente de la Capa de Acceso y funciona como el límite de enrutamiento entre las VLANs (enrutamiento inter-VLAN). Es responsable de implementar políticas de red (como ACLs) y de la agregación de enlaces (Ghafoor et al., 2018). En este edificio, se utiliza un diseño de \textbf{Núcleo Colapsado (Collapsed Core)}, donde las funciones de Distribución y Núcleo de la LAN son consolidadas en un único switch multicapa (`SW-Core-P1`).
		
		\item \textbf{Capa de Núcleo (Verde):} Proporciona el backbone de alta velocidad de la red. Su único propósito es conmutar paquetes tan rápido como sea posible, garantizando alta disponibilidad y fiabilidad. En este diseño, la capa de núcleo está representada por los routers (`ROUTER-P1` y `ROUTER-INNOVACION`), que manejan el enrutamiento de alta velocidad entre edificios y hacia redes externas (Odom, 2019).
	\end{itemize}