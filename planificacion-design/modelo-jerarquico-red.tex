
\subsubsection{Modelo jerárquico de red}

El diseño propuesto se fundamenta en el modelo jerárquico de tres capas, una arquitectura ampliamente adoptada por su escalabilidad, rendimiento y facilidad de gestión (Odom, 2019).

\begin{figure}[H]
  \centering
  \includegraphics[width=0.95\linewidth]{diagrama-red-jerarquico}
  \caption{\textit{Modelo jerárquico de la red propuesta}}
  \label{fig:modelo-jerarquico}
\end{figure}

El modelo divide la red lógicamente en tres capas funcionales:
	
	\begin{itemize}
		\item \textbf{Capa de acceso (Morado)}. Punto de entrada para dispositivos finales como equipos de cómputo, teléfonos VoIP, cámaras CCTV y puntos de acceso. Proporciona conectividad y aplica políticas a nivel de puerto (Kurose \& Ross, 2021). Los switches de los pisos 2 y 3, así como los puertos de usuario del piso 1, operan en esta capa.
		\item \textbf{Capa de distribución (Azul)}. Agrega el tráfico de la capa de acceso y funge como límite de enrutamiento entre VLAN (enrutamiento inter-VLAN). Implementa políticas de red y agregación de enlaces (Ghafoor et al., 2018). Se adopta un núcleo colapsado en el switch multicapa central.
		\item \textbf{Capa de núcleo (Verde)}. Proporciona el backbone de alta velocidad conmutando paquetes de forma eficiente y garantizando alta disponibilidad. En este diseño la capa está representada por los routers que enlazan el edificio con redes externas (Odom, 2019).
	\end{itemize}