Para el apartado de seguridad, se utilizará una computadora de propósito general, adaptada para tener las suficientes interfaces para conectarse a la red, corriendo el sistema operativo \textit{OPNsense}, el cual es de código abierto, y reduce el costo de pagar licencias de software de código cerrado.

OPNsense es un sistema operativo basado en FreeBSD la cual está pensada para ser utilizada en dispositivos de red como firewalls y routers, lo cual da una facilidad para acceder a herramientas comunes al momento de configurar el equipo, como lo puede ser tablas de reglas de firewall, redirección de puertos, y, al ser un sistema operativo de código abierto, da puerta a poder configurarlo a las necesidades de la red.
\begin{figure}
  \caption{Captura de pantalla de OPNsense}
	\begin{center}
	  \includegraphics[width=0.6\textwidth]{images/opnsense-screenshot.png}
	\end{center}
\end{figure}
Se utilizará como equipo el Protectli Vault 4 Port, con el cual, un equipo de bajo consumo de energía, y, que si en un futuro se necesita otro software de firewall, puede ser fácilmente configurado. Este será añadido a la cotización
