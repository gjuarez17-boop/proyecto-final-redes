\subsection{Solución de Firewall (Hardware y Software)}
Los firewalls constituyen un componente esencial en la seguridad de redes moderna. Actúan como una barrera de defensa fundamental, cuya función principal es supervisar, filtrar y controlar todo el tráfico de red entrante y saliente basándose en un conjunto de reglas de seguridad predefinidas. Su objetivo es identificar y bloquear el tráfico no deseado o malicioso antes de que pueda comprometer los activos internos. De esta manera, los firewalls se convierten en una de las herramientas de protección de información con mayor relevancia, ayudando a preservar los tres pilares de la seguridad de la información: la integridad, la confidencialidad y la disponibilidad \parencite{MoraBandera2020}

La solución propuesta se compone de una plataforma de hardware dedicada y un sistema operativo de firewall de nueva generación.

\subsubsection{Plataforma de Hardware}
El hardware seleccionado es el \textbf{Protectli Vault de cuatro puertos}. Este equipo es un dispositivo de hardware dedicado (tipo "barebones" o mini-PC) diseñado específicamente para ejecutar software de seguridad de red de forma eficiente. A diferencia de un router de consumo, está optimizado para el alto rendimiento en tareas de enrutamiento y cifrado.

La justificación de esta elección se basa en varios puntos clave:
\begin{itemize}
    \item \textbf{Rendimiento de Red:} Utiliza procesadores Intel\textregistered{} de múltiples núcleos y, crucialmente, \textbf{puertos de red (NICs) Intel\textregistered{}}, que garantizan un rendimiento y compatibilidad superiores para el manejo de tráfico intenso y la segmentación por VLAN.
    \item \textbf{Eficiencia y Fiabilidad:} Su diseño sin ventilador ("fanless") asegura un funcionamiento 24/7 completamente silencioso y un bajo consumo energético.
    \item \textbf{Soporte VPN de alta velocidad:} Incluye soporte de hardware \textbf{AES-NI}, lo cual permite al procesador acelerar el cifrado y descifrado. Esto es fundamental para procesar túneles VPN (IPsec, OpenVPN) de alta velocidad sin sobrecargar la CPU.
    \item \textbf{Flexibilidad de Software:} Al ser un hardware "agnóstico", valida la estrategia de "futuras migraciones de software sin cambios físicos", evitando la dependencia de un solo proveedor (vendor lock-in).
\end{itemize}

\subsubsection{Plataforma de Software (NGFW)}
El sistema operativo que se ejecutará sobre el hardware Protectli Vault es \textbf{\textit{OPNsense}}. Esta es una solución de firewall de código abierto basada en FreeBSD, lo que elimina completamente los costos de licenciamiento de software propietario.

\textit{OPNsense} está orientado a dispositivos de red y proporciona una plataforma robusta para construir un \textbf{Firewall de Nueva Generación (NGFW)}.

La justificación para la elección de OPNsense complementa al hardware:
\begin{itemize}
    \item \textbf{Capacidades NGFW Integradas:} Más allá de la inspección con estado (stateful inspection), OPNsense incluye de forma nativa (o con plugins) un Sistema de Prevención de Intrusos (IPS) basado en Suricata, filtrado de aplicaciones y capacidades de proxy web.
    \item \textbf{Gestión Centralizada:} Como se observa en la Figura \ref{fig:opnsense}, proporciona una interfaz web moderna y fácil de usar para la gestión de todas las reglas, servicios de red (DHCP, DNS), VPNs y monitoreo de tráfico.
    \item \textbf{Seguridad y Actualizaciones:} Cuenta con un ciclo de desarrollo activo, ofreciendo actualizaciones de seguridad semanales, lo que garantiza una rápida respuesta a nuevas vulnerabilidades.
\end{itemize}

\begin{figure}[H]
\centering
\includegraphics[width=0.6\textwidth]{opnsense-screenshot}
\caption{\textit{Interfaz de administración de OPNsense}}\label{fig:opnsense}
\begin{flushleft}
    	\textit{Nota.} Imagen de referencia ilustrativa de la interfaz de OPNsense; no corresponde a una captura propia del proyecto ni refleja configuraciones reales del entorno diseñado.
\end{flushleft}
\end{figure}