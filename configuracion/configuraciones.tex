\subsection{Configuraciones de equipos activos}

El archivo de Packet Tracer disponible en \href{https://drive.google.com/file/d/1mHfBjxOKNXVSxJJW_NH12IAQf1O9go8d/view?usp=sharing}{este enlace} contiene la implementación lógica de la red. Las principales configuraciones aplicadas son las siguientes:

\begin{itemize}
  \item Configuraciones básicas. Cada switch y router cuenta con un \texttt{hostname} único y contraseñas cifradas para acceso por consola, VTY y modo privilegiado.
  \item Configuración de VLAN. Se crean las VLAN 10 (ADMIN), 20 (ALUMNOS), 30 (PROFESORES), 40 (VOIP) y 50 (CÁMARAS) en el switch central. Los puertos de acceso se asignan a su VLAN correspondiente y los enlaces entre switches se configuran como \texttt{trunk}.
  \item Telefonía IP. Los teléfonos se ubican en la VLAN 40 con extensiones únicas y se verifica la comunicación entre ellos.
  \item Videovigilancia. Las cámaras CCTV emplean direcciones estáticas dentro del rango de la VLAN 50 para garantizar supervisión permanente.
  \item Direccionamiento IPv4. El plan VLSM define los gateways configurados en el switch multicapa del piso 1 para permitir el enrutamiento inter-VLAN.
\end{itemize}