\subsection{Segmentación de VLAN}

\begin{figure}[H]
  \centering
  \includegraphics[width=0.9\textwidth]{diagrama-logico-red-vlans}
  \caption{\textit{Diagrama lógico de segmentación por VLAN}}
  \label{fig:diagrama-logico}
  \begin{flushleft}
    	\textit{Nota.} El diagrama codifica por colores la segmentación implementada por VLANS.
  \end{flushleft}
\end{figure}

Para optimizar el espacio de direccionamiento se aplica VLSM sobre el bloque base 148.220.0.0/21, permitiendo subredes acordes con la densidad de dispositivos (Singh \& Kaur, 2014).

Las VLAN de usuarios (20 Alumnos, 30 Profesores) operan mediante DHCP para asignación automática, mientras que las VLAN de infraestructura (10 Admin, 40 VoIP, 50 Cámaras) emplean direcciones estáticas. La primera IP utilizable de cada subred se reserva como gateway en el switch central para habilitar enrutamiento inter-VLAN.