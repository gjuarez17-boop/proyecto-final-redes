\documentclass[stu, 12pt]{apa7}
\usepackage{mathptmx}
\usepackage[spanish,mexico]{babel}
\usepackage{bookmark}
\usepackage[style=apa]{biblatex}
\usepackage{csquotes}
\usepackage{hyperref}
\usepackage{longtable}
\usepackage{booktabs}
\usepackage{tabularx}
\usepackage{graphicx}
\usepackage{geometry}
\usepackage{caption}
\usepackage{subcaption}  
\usepackage{array}
\usepackage[colorlinks=true, urlcolor=blue]{hyperref} 
% --- Configuración de captions estilo APA 7 ---
\captionsetup{
	justification=justified,
	singlelinecheck=false,
	labelfont=bf,
	labelsep=period
}
% Configuración específica para las sub-figuras
\captionsetup[subfigure]{
	labelfont=rm, % Letra normal para (a), (b)
	labelformat=simple % Formato simple (a, b)
}
\addbibresource{main-sources.bib}
\hypersetup{
  colorlinks=true,
  urlcolor=blue,
  linkcolor=black
}

\newcolumntype{L}{>{\raggedright\arraybackslash}X}

\title{Red de edificio de Aulas}
\authorsnames{Cárdenas Camacho Alexis -- 295592, Juárez Ramírez Gabriel -- 325846, Macías Fonseca Alejandro -- 325724, Mata Guerra David -- 325797}
\authorsaffiliations{Facultad de Informática, Universidad Autónoma de Querétaro}
\course{Diseño y Soporte de Redes}
\professor{Estévez Serrato César}
\duedate{28 de noviembre de 2025}

\begin{document}
\maketitle
\tableofcontents
\pagebreak
\section{Planificación y Diseño}
\documentclass[letterpaper, 12pt]{article}
\usepackage[utf8]{inputenc}
\usepackage[T1]{fontenc}
\usepackage{lmodern} 
\usepackage{times} 
\usepackage[spanish]{babel}
\usepackage[margin=2.5cm]{geometry}
\usepackage{setspace}
\doublespacing
\usepackage{indentfirst}
\setlength{\parindent}{0.5in} 
\usepackage[hidelinks]{hyperref}



\title{Proyecto Final: Diseño y Soporte de Redes \\ Edificio de salones (A)}
\author{Equipo 2}
\date{\today}

%%%%%%%%%%%%%%%%%%%%%%%%%%%%%%%%%%%%%%%%%%%%%%%%%%%%%%%%%%%%%%%%%%%%%%%%%%%%%%%
% INICIO DEL DOCUMENTO
%%%%%%%%%%%%%%%%%%%%%%%%%%%%%%%%%%%%%%%%%%%%%%%%%%%%%%%%%%%%%%%%%%%%%%%%%%%%%%%
\begin{document}
	
	% Descomente la siguiente línea si quiere una portada
	% \maketitle
	
	% Use \section{}, \subsection{}, etc. para estructurar
	\section{Fase 1: Planeación y Diseño}
	
	\subsection{Selección de Cableado}
	
	Para el diseño de la red del \textbf{Edificio de salones (A)}, se ha seleccionado una arquitectura de cableado estructurado que prioriza el requisito de una \textbf{red ``muy estable''}, respetando al mismo tiempo la limitación de \textbf{``Bajo presupuesto''}. La selección se divide en tres segmentos principales:
	
	\subsubsection{Cableado Horizontal}
	
	\begin{itemize}
		\item \textbf{Cable Seleccionado:} Cable de Par Trenzado No Blindado (UTP) \textbf{Categoría 6 (Cat 6)}.
		
		\item \textbf{Justificación:}
		\begin{itemize}
			\item \textbf{Costo-Beneficio:} Cat 6 ofrece el mejor balance entre rendimiento y costo para un presupuesto ajustado, superando a Cat 5e en fiabilidad a largo plazo.
			\item \textbf{Ancho de Banda:} Proporciona un ancho de banda de 1 Gbps, más que suficiente para cumplir con el requisito de ``un nodo por cuarto''.
			\item \textbf{Soporte PoE:} Es fundamental para la alimentación eficiente de los dispositivos IP requeridos, como las ``4 cámaras por piso (CCTV)'' y los ``teléfono[s] VoIP'' de las oficinas centrales, eliminando la necesidad de fuentes de alimentación eléctrica individuales.
		\end{itemize}
	\end{itemize}
	
	\subsubsection{Cableado Vertical (Backbone del Edificio)}
	
	\begin{itemize}
		\item \textbf{Cable Seleccionado:} Fibra Óptica \textbf{Duplex Multimodo OM3} (50/125 $\mu$m).
		
		\item \textbf{Justificación:}
		\begin{itemize}
			\item \textbf{Alta Estabilidad y Velocidad:} Para garantizar una red ``muy estable'', el backbone que interconecta los switches de cada piso debe ser de alta capacidad. La fibra OM3 permite enlaces de 10 Gbps (e incluso 40 Gbps) en distancias cortas, eliminando cuellos de botella.
			\item \textbf{Inmunidad a Interferencia:} Al ser fibra óptica, es completamente inmune a la interferencia electromagnética (EMI), lo que es vital en un edificio con múltiples instalaciones eléctricas.
		\end{itemize}
	\end{itemize}
	
	\subsubsection{Conexión Inter-edificio (Enlace a Edificio de Innovación)}
	
	\begin{itemize}
		\item \textbf{Cable Seleccionado:} Fibra Óptica \textbf{Duplex Multimodo OM3, de tipo Armado} para exteriores.
		
		\item \textbf{Justificación:}
		\begin{itemize}
			\item \textbf{Cumplimiento de Requisito:} Satisface la solicitud de ``Hacer una conexión entre el edificio de salones y el edificio de innovación''.
			\item \textbf{Protección Física:} Se ha seleccionado una versión \textbf{armada} para proteger el enlace contra daños físicos, ya sea en canalización subterránea o aérea. Esta armadura lo protege de la compresión, la tensión y los roedores, asegurando la longevidad de la inversión.
			\item \textbf{Rendimiento:} La fibra OM3 es adecuada para este enlace, asumiendo que la distancia al Edificio de Innovación es inferior a 300 metros para un enlace de 10 Gbps.
		\end{itemize}
	\end{itemize}
	
	
\end{document}

\subsection{Equipos}
\subsubsection{Equipo activo}
\subsubsection{Equipo activo}

La infraestructura activa comprende los equipos electrónicos encargados de la transmisión, procesamiento y encaminamiento de la información dentro de la red. Estos dispositivos incluyen routers, switches, antenas y servidores de control y gestión \parencite{martinez2020infraestructura}.

Los equipos activos poseen componentes electrónicos propios, como procesadores y memoria, que permiten gestionar el tráfico, optimizar la velocidad de transmisión y simplificar la administración \parencite{newserverlife_network_equipment_concept_and_types_2023}. En este proyecto se agrupan en categorías de conmutación, enrutamiento, seguridad, comunicación inalámbrica y gestión. La Tabla \ref{tab:equipos_activos} muestra los principales dispositivos contemplados.

\begin{table}[H]
	\caption{\textit{Equipos activos utilizados en la red del proyecto}}
	\label{tab:equipos_activos}
	\centering
	\begin{tabularx}{0.8\textwidth}{lX}
			\toprule
			\textbf{Categoría de equipo activo} & \textbf{Equipos} \\
		\midrule
		Dispositivos de conmutación & Switches de acceso y núcleo. \\
		Dispositivos de encaminamiento & Routers para la salida perimetral. \\
		Dispositivos de acceso inalámbrico & Puntos de acceso con soporte PoE. \\
		Dispositivos de seguridad & Firewalls perimetrales y locales. \\
		Dispositivos de comunicación de voz & Teléfonos VoIP administrados. \\
		\bottomrule
	\end{tabularx}
\end{table}

\subsubsection{Equipo pasivo}
\subsubsection{Equipo pasivo}

La infraestructura pasiva reúne los componentes físicos que soportan y conectan a los equipos activos. Aunque no procesan datos, son esenciales para transmitir la información de forma estable y confiable (Cisneros, 2021).

Los elementos principales incluyen el cableado estructurado, conectores, paneles de parcheo y la infraestructura de soporte. El cableado horizontal se compone de cable UTP categoría 6, encargado de enlazar los nodos con los armarios de comunicaciones, mientras que el backbone recurre a fibra óptica multimodo OM3 para enlaces de alta capacidad (SYSTIMAX, 2020). La organización se completa con \textit{patch panels}, racks y accesorios para mantener el orden del tendido (Fluke Networks, 2022).

\begin{table}[H]
	\caption{\textit{Componentes pasivos utilizados en la red del proyecto}}
	\label{tab:equipos_pasivos}
	\centering
	\begin{tabularx}{0.95\textwidth}{lX}
			oprule
			extbf{Categoría de equipo pasivo} & \textbf{Componentes} \\
		\midrule
		Cableado de cobre & Cableado UTP Cat 6 y conectores RJ-45 de alto desempeño. \\
		Cableado de fibra óptica & Fibra multimodo OM3 con terminaciones reforzadas. \\
		Gestión de cableado & Canaletas, bandejas y organizadores modulares. \\
		Terminación de red & \textit{Patch panels} y \textit{faceplates} identificados. \\
		Infraestructura física & Racks autoportantes y gabinetes de telecomunicaciones. \\
		\bottomrule
	\end{tabularx}
\end{table}
\subsection{Metodología}
\textbf{Metodología}
El diseño de la red hipotética para el Edificio de salones (A)  se gestionará utilizando la metodología del ciclo de vida de la red PPDIOO (Preparar, Planificar, Diseñar, Implementar, Operar y Optimizar). Esta metodología, vista en clase, proporciona un marco estructurado que asegura que se cumplan todos los requisitos técnicos y operativos del proyecto.

El objetivo principal es usar este modelo para ejecutar de manera ordenada las seis fases de entrega definidas en el documento del proyecto. La correspondencia entre PPDIOO y las fases del proyecto será la siguiente:

Fase 1: Planeación y diseño  Esta fase se cubre con las tres primeras etapas de PPDIOO:

Preparar (Prepare): Se definen los objetivos del proyecto. En nuestro caso: diseñar la red del Edificio A con "Bajo presupuesto" (\$490,200) , asegurar conectividad con el Edificio de Innovación , implementar CCTV , VoIP y una red inalámbrica "muy estable".

Planificar (Plan): Se identifican los recursos, se crea el cronograma de actividades y se detallan los costos de material y mano de obra para la cotización, asegurando no exceder el presupuesto.

Diseñar (Design): Se crea la solución técnica detallada, incluyendo la topología de red, selección de equipo activo y pasivo (Cat 6 y Fibra OM3), el diagrama de red, la tabla de direccionamiento IP y el diseño de la WLAN segmentada (alumnos/docentes).

Fase 2: Implementación de la red (configuraciones lógicas)  Corresponde a la etapa Implementar (Implement) de PPDIOO. En este punto, se llevarán a cabo las configuraciones lógicas de los switches, routers y puntos de acceso (APs) según lo establecido en la fase de Diseño.

Fase 4: Implementación de seguridad  Esta fase también es parte de la etapa Implementar (Implement). Aquí se configuran las políticas de seguridad de la red, como la segmentación de la WLAN, la creación de VLANs (para alumnos, docentes, VoIP, CCTV) y las listas de control de acceso (ACLs) para proteger la red.

Fase 3: Monitoreo y análisis de la red y Fase 5: Administración de fallas  Ambas fases del proyecto se engloban en la etapa Operar (Operate) de PPDIOO. Esta etapa se enfoca en la gestión diaria de la red, lo que incluye:

El monitoreo constante para establecer un rendimiento base y detectar anomalías (Monitoreo y análisis).

La detección, diagnóstico y resolución proactiva de problemas (Administración de fallas) para mantener la red "muy estable".

Fase 6: Calidad en el servicio  Esta fase final corresponde a la etapa Optimizar (Optimize) de PPDIOO. Una vez que la red está operativa y monitoreada, se implementarán políticas de Calidad de Servicio (QoS) para priorizar el tráfico sensible, como las llamadas de los teléfonos VoIP y el video de las cámaras CCTV, asegurando su correcto funcionamiento incluso bajo carga de red.



El uso de PPDIOO como metodología de trabajo nos asegura que cada entregable del proyecto  sea un resultado lógico del ciclo de vida de la red.


\subsection{Tabla de Direccionamiento IP}
\begin{table}[h]
	\centering

	\caption{\textit{Tabla de Direccionamiento IP (VLSM)}}
	\label{tab:direccionamiento_ip}
	\begin{tabularx}{\textwidth}{l l l l l l X} 
		\toprule

		VLAN & Nombre & Asignación & Subred & Prefijo & Gateway & Rango Utilizable \\
		\midrule
		20 & ALUMNOS & DHCP & 148.220.0.0 & /22 & 148.220.0.1 & Pool: 148.220.0.2 - 148.220.3.254 \\
		30 & PROFESORES & DHCP & 148.220.4.0 & /25 & 148.220.4.1 & Pool: 148.220.4.2 - 148.220.4.126 \\
		40 & VOIP & Estática & 148.220.4.128 & /25 & 148.220.4.129 & Rango: 148.220.4.130 - 148.220.4.254 \\
		50 & CAMARAS & Estática & 148.220.5.0 & /25 & 148.220.5.1 & Rango: 148.220.5.2 - 148.220.5.126 \\
		10 & ADMIN & Estática & 148.220.5.128 & /26 & 148.220.5.129 & Rango: 148.220.5.130 - 148.220.5.190 \\
		\bottomrule
	\end{tabularx}
	
	\parbox{\textwidth}{
		\textit{Nota.} El plan utiliza VLSM para optimizar el bloque de red `148.220.0.0/21`. La puerta de enlace (gateway) es la primera IP utilizable de cada subred. Las VLANs de infraestructura (ADMIN, VOIP, CAMARAS) usan asignación estática, mientras que las VLANs de usuarios (ALUMNOS, PROFESORES) usan DHCP.
	}
\end{table}
\subsection{Tamaño de Red}
\subsection{Tamaño de red}

El proyecto contempla la implementación de una red de área local (LAN) cableada y segmentada mediante VLAN dinámicas (VLSM) para separar los servicios de telefonía IP, videovigilancia, alumnos, docentes y administración.

El backbone de fibra óptica multimodo OM3 interconecta los armarios de telecomunicaciones entre pisos y proporciona el enlace hacia el edificio de Innovación, extendiendo la infraestructura a un entorno de campus. El dimensionamiento considera nodos cableados, cámaras IP, teléfonos VoIP, equipos de red y puntos de acceso inalámbrico, con capacidad de crecimiento proyectada a 20 años.

Los elementos principales de la red se resumen en los siguientes puntos:

\begin{itemize}
  \item Tipo de red: LAN cableada (Cat 6 UTP) con integración de WLAN para usuarios móviles.
  \item Topología lógica: estrella jerárquica con switches de acceso y un switch central.
  \item Enlace interedificio: fibra óptica multimodo OM3 hacia el edificio de Innovación.
  \item Segmentación lógica: VLAN por servicio con soporte de calidad de servicio para voz.
\end{itemize}

El bloque base 148.220.0.0/21 permite 2046 hosts útiles por VLAN. Para optimizar su uso, las VLAN se subdividen con VLSM según la densidad de dispositivos de cada servicio y los requisitos de crecimiento.

\begin{table}[H]
  \caption{\textit{Resumen general de la red implementada}}
  \label{tab:resumen-red}
  \centering
  \begin{tabularx}{\textwidth}{l l l l X}
    	\toprule
    	\textbf{Tipo de red} & \textbf{Medio físico} & \textbf{Velocidad} & \textbf{Backbone} & \textbf{Segmentación} \\
    \midrule
    LAN cableada & Cobre Cat 6 UTP y fibra OM3 & 1 Gbps (cobre); 10 Gbps (fibra) & Fibra multimodo entre edificios & VLAN por servicio\\
    \bottomrule
  \end{tabularx}
\end{table}
\subsection{Diagramas de Red}
\subsubsection{Modelo jerárquico de red}

\subsubsection{Modelo jerárquico de red}

El diseño propuesto se fundamenta en el modelo jerárquico de tres capas, una arquitectura ampliamente adoptada por su escalabilidad, rendimiento y facilidad de gestión (Odom, 2019).

\begin{figure}[H]
  \centering
  \includegraphics[width=0.95\linewidth]{diagrama-red-jerarquico}
  \caption{\textit{Modelo jerárquico de la red propuesta}}
  \label{fig:modelo-jerarquico}
\end{figure}

El modelo divide la red lógicamente en tres capas funcionales:
	
	\begin{itemize}
		\item \textbf{Capa de acceso (Morado)}. Punto de entrada para dispositivos finales como equipos de cómputo, teléfonos VoIP, cámaras CCTV y puntos de acceso. Proporciona conectividad y aplica políticas a nivel de puerto (Kurose \& Ross, 2021). Los switches de los pisos 2 y 3, así como los puertos de usuario del piso 1, operan en esta capa.
		\item \textbf{Capa de distribución (Azul)}. Agrega el tráfico de la capa de acceso y funge como límite de enrutamiento entre VLAN (enrutamiento inter-VLAN). Implementa políticas de red y agregación de enlaces (Ghafoor et al., 2018). Se adopta un núcleo colapsado en el switch multicapa central.
		\item \textbf{Capa de núcleo (Verde)}. Proporciona el backbone de alta velocidad conmutando paquetes de forma eficiente y garantizando alta disponibilidad. En este diseño la capa está representada por los routers que enlazan el edificio con redes externas (Odom, 2019).
	\end{itemize}
\subsubsection{Diagrama representativo}

% Buscar imágenes en la carpeta ../images (carpeta hermana a planificacion-design)
\graphicspath{{../images/}}
		extbf{Diagrama de la Red Física}
	
	La \textbf{Figura 1} muestra la topología física de la red propuesta para el edificio administrativo de tres niveles. En él se representan los equipos de cómputo, switches, routers, puntos de acceso y servidores distribuidos por planta, así como el cableado estructurado correspondiente. Este diseño busca garantizar la conectividad mediante una arquitectura jerárquica eficiente y escalable.
	
	La estructura detallada de cada rack se encuentra especificada en las \textbf{Figuras [X]} y \textbf{[Y]} de los diagramas de rack.
	
	\begin{figure}[h!]
		\centering
		\begin{subfigure}[b]{0.3\textwidth}
			\centering
			\includegraphics[width=\linewidth]{cisco-diagram-piso1}
			\caption{Piso 1}
			\label{fig:piso1}
		\end{subfigure}
		\hfill 
		\begin{subfigure}[b]{0.3\textwidth}
			\centering
			\includegraphics[width=\linewidth]{cisco-diagram-piso2}
			\caption{Piso 2}
			\label{fig:piso2}
		\end{subfigure}
		\hfill 
		\begin{subfigure}[b]{0.3\textwidth}
			\centering
			\includegraphics[width=\linewidth]{cisco-diagram-piso3}
			\caption{Piso 3}
			\label{fig:piso3}
		\end{subfigure}
		

		\caption{\textit{Topología Física de la Red del Edificio Administrativo}}
		\label{fig:red-fisica}
		
		\parbox{\textwidth}{
			\vspace{2mm}
			\small
			\textit{Nota.} Los paneles (a), (b), y (c) muestran la distribución de equipos por piso. Cada piso cuenta con su propio switch de acceso conectado a un switch de distribución central ubicado en el piso 1 (Este cumple con la función tanto core como acceso al piso 1). Esta estructura permite facilitar el mantenimiento de los equipos. Planeado para una escalabilidad a 20 años aproximadamente.
		}
	\end{figure}
\subsubsection{Diagrama físico estandarizado}
\input{planificacion-design/diagrama-ansi.tex}
\subsubsection{Diagrama de Rack}

\subsubsection{Diagrama de rack}

\begin{figure}[H]
  \centering
  \includegraphics[width=0.45\linewidth]{Diagrama rack piso 1(1)}
  \caption{\textit{Distribución del rack principal en el piso 1}}
  \label{fig:rack_piso1}
\end{figure}

\begin{table}[H]
  \caption{\textit{Componentes del gabinete central (piso 1)}}
  \label{tab:rack_piso1}
  \centering
  \begin{tabularx}{\textwidth}{l l l X}
    	\toprule
    	\textbf{Unidad (U)} & \textbf{Dispositivo} & \textbf{Tipo} & \textbf{Función o descripción} \\
    \midrule
    1U & Patch panel modular & Pasivo & Termina el cableado proveniente de los pisos superiores. \\
    2U & Organizador de cable & Pasivo & Ordena los conductores entre el panel y los equipos activos. \\
    1U & Switch core & Activo & Conecta los switches de cada piso y realiza enrutamiento interno. \\
    1U & Router core & Activo & Gestiona la salida a redes externas y la interconexión al campus. \\
    3U & UPS & Activo & Proporciona respaldo energético ante fallas eléctricas. \\
    1U & Módulo de enfriamiento & Activo & Asegura ventilación y control térmico dentro del gabinete. \\
    1U & PDU & Pasivo & Distribuye energía hacia los dispositivos montados. \\
    \bottomrule
  \end{tabularx}
\end{table}

\begin{table}[H]
  \caption{\textit{Componentes auxiliares para el rack del piso 1}}
  \label{tab:bom_piso1}
  \centering
  \begin{tabularx}{\textwidth}{r l X}
    	\toprule
    	\textbf{Cantidad} & \textbf{Unidad} & \textbf{Componente} \\
    \midrule
    \multicolumn{3}{l}{\textit{Acomodo (rack)}} \\
    1 & Pza & Gabinete de pared de 12U. \\
    1 & Pza & UPS de respaldo. \\
    1 & Pza & PDU horizontal de 1U. \\
    1 & Pza & Organizador de cables horizontal de 1U. \\
    1 & Paq & Tornillería y tuercas para montaje. \\
    \multicolumn{3}{l}{\textit{Conexión (cableado)}} \\
    1 & Pza & Patch panel modular para fibra y cobre. \\
    2 & Pza & Adaptadores de fibra LC dúplex multimodo. \\
    26 & Pza & Patch cords de cobre categoría 6. \\
    2 & Pza & Patch cords de fibra multimodo. \\
    \multicolumn{3}{l}{\textit{Equipo activo}} \\
    1 & Pza & Router. \\
    1 & Pza & Switch de 24 puertos con ranuras SFP. \\
    2 & Pza & Módulos SFP 1 Gbps multimodo. \\
    \bottomrule
  \end{tabularx}
\end{table}

\begin{figure}[H]
  \centering
  \includegraphics[width=0.5\linewidth]{diagrama_rack piso 2}
  \caption{\textit{Distribución del rack de telecomunicaciones en el piso 2}}
  \label{fig:rack_piso2}
\end{figure}

\begin{table}[H]
  \caption{\textit{Componentes instalados en el rack del piso 2}}
  \label{tab:rack_piso2}
  \centering
  \begin{tabularx}{\textwidth}{l l l X}
    	\toprule
    	\textbf{Unidad (U)} & \textbf{Dispositivo} & \textbf{Tipo} & \textbf{Función o descripción} \\
    \midrule
    1U & Módulo de enfriamiento & Activo & Mantiene ventilación y control térmico del gabinete. \\
    1U & Patch panel modular & Pasivo & Termina el cableado horizontal de la planta. \\
    2U & Organizador de cables & Pasivo & Ordena los patch cords entre panel y switch. \\
    1U & Switch de acceso & Activo & Conecta los dispositivos finales del piso. \\
    3U & UPS & Activo & Provee respaldo de energía. \\
    1U & PDU & Pasivo & Distribuye energía hacia el equipo activo. \\
    \bottomrule
  \end{tabularx}
\end{table}

\begin{table}[H]
  \caption{\textit{Componentes auxiliares para la instalación del rack del piso 2}}
  \label{tab:bom_piso2}
  \centering
  \begin{tabularx}{\textwidth}{r l X}
    	\toprule
    	\textbf{Cantidad} & \textbf{Unidad} & \textbf{Componente} \\
    \midrule
    \multicolumn{3}{l}{\textit{Acomodo (rack)}} \\
    1 & Pza & Gabinete de pared de 12U. \\
    1 & Pza & Módulo de enfriamiento de 1U. \\
    1 & Pza & Organizador de cables horizontal de 2U. \\
    1 & Pza & UPS de 3U. \\
    1 & Pza & PDU horizontal de 1U. \\
    1 & Paq & Tornillería y herrajes de montaje. \\
    \multicolumn{3}{l}{\textit{Conexión (cableado)}} \\
    1 & Pza & Patch panel modular para fibra y cobre. \\
    24 & Pza & Patch cords de cobre categoría 6. \\
    1 & Pza & Patch cord de fibra multimodo. \\
    \multicolumn{3}{l}{\textit{Equipo activo}} \\
    1 & Pza & Switch de 24 puertos GbE con 2 SFP. \\
    2 & Pza & Módulos SFP multimodo LC de 1 Gbps. \\
    \bottomrule
  \end{tabularx}
\end{table}

\begin{figure}[H]
  \centering
  \includegraphics[width=0.5\linewidth]{diagrama_rack_piso 3}
  \caption{\textit{Distribución del rack de telecomunicaciones en el piso 3}}
  \label{fig:rack_piso3}
\end{figure}

\begin{table}[H]
  \caption{\textit{Componentes instalados en el rack del piso 3}}
  \label{tab:rack_piso3}
  \centering
  \begin{tabularx}{\textwidth}{l l l X}
    	\toprule
    	\textbf{Unidad (U)} & \textbf{Dispositivo} & \textbf{Tipo} & \textbf{Función o descripción} \\
    \midrule
    1U & Módulo de enfriamiento & Activo & Mantiene la temperatura operativa del gabinete. \\
    1U & Patch panel modular & Pasivo & Termina el cableado horizontal del piso. \\
    2U & Organizador de cables & Pasivo & Ordena los patch cords hacia el switch. \\
    1U & Switch de acceso & Activo & Conecta los equipos finales del nivel. \\
    3U & UPS & Activo & Proporciona respaldo de energía. \\
    1U & PDU & Pasivo & Distribuye energía hacia los equipos. \\
    \bottomrule
  \end{tabularx}
\end{table}

\begin{table}[H]
  \caption{\textit{Componentes auxiliares para el rack del piso 3}}
  \label{tab:bom_piso3}
  \centering
  \begin{tabularx}{\textwidth}{r l X}
    	\toprule
    	\textbf{Cantidad} & \textbf{Unidad} & \textbf{Componente} \\
    \midrule
    \multicolumn{3}{l}{\textit{Acomodo (rack)}} \\
    1 & Pza & Gabinete de pared de 12U. \\
    1 & Pza & Módulo de enfriamiento de 1U. \\
    1 & Pza & Organizador de cables horizontal de 2U. \\
    1 & Pza & UPS de 3U. \\
    1 & Pza & PDU horizontal de 1U. \\
    1 & Paq & Tornillería y herrajes de sujeción. \\
    \multicolumn{3}{l}{\textit{Conexión (cableado)}} \\
    1 & Pza & Patch panel modular para fibra y cobre. \\
    24 & Pza & Patch cords de cobre categoría 6. \\
    1 & Pza & Patch cord de fibra multimodo. \\
    \multicolumn{3}{l}{\textit{Equipo activo}} \\
    1 & Pza & Switch de 24 puertos GbE con 2 SFP. \\
    2 & Pza & Módulos SFP multimodo LC de 1 Gbps. \\
    \bottomrule
  \end{tabularx}
\end{table}
\subsection{Topología de Red}
\input{planificacion-design/topologia.tex}
\subsection{Cotización}
Una vez listados los equipos necesarios, se listó la cotización de los equipos en la tabla %\ref{table:cotizacion}
\begin{longtable}[c]{p{6cm}crr}
  %\label{table:cotizacion}
  \caption{Tabla de cotizaciones}\\
      \hline
      \textbf{Material} & \textbf{Cantidad} & \textbf{Precio Unitario} & \textbf{Subtotal}\\
      \hline
      \endfirsthead
      \hline
      \textbf{Material} & \textbf{Cantidad} & \textbf{Precio Unitario} & \textbf{Subtotal}\\
      \hline
      \endhead
      \endfoot
      \hline
      \textbf{Total} & & & \$261,437.90 MXN\\
      \hline
      \endlastfoot
      \href{https://www.cyberpuerta.mx/Computo-Hardware/Redes/Switches/Switch-TP-Link-Gigabit-Ethernet-TL-SG2428P-24-Puertos-PoE-10-100-1000Mbps-4-Puertos-SFP-56-Gbit-s-8000-Entradas-Administrable.html}{Switch TP-Link TL-SG2428P 24 UTP 4 SFP PoE+ 250W} & 3 & \$5,549 MXN & \$16,647 MXN\\
      \href{https://www.cyberpuerta.mx/Computo-Hardware/Servidores/Accesorios-para-Servidores/Racks-y-Gabinetes/Enson-Gabinete-para-Pared-19-ENS-RKGB12U-12U-hasta-60kg.html}{Gabinete Enson ENS-RKGB12U} & 1 & \$2,949 MXN & \$2,949 MXN\\
      \href{https://www.cyberpuerta.mx/Computo-Hardware/Servidores/Accesorios-para-Servidores/Racks-y-Gabinetes/LinkedPRO-Rack-de-Pared-Abierto-de-19-8U-Negro.html}{LinkedPRO Rack de Pared 8U} & 2 & \$1,199 MXN & \$2,398 MXN\\
      \href{https://www.cyberpuerta.mx/Computo-Hardware/Servidores/Accesorios-para-Servidores/Cableado-Estructurado-para-Servidores/Paneles-de-Parcheo/Intellinet-Panel-de-Parcheo-Cat6-24-Puertos-RJ-45-1U-Negro.html}{Patch Panel Intellinet 24 puertos Cat6 1U} & 3 & \$749 MXN & \$2,247 MXN\\ 
      \href{https://www.cyberpuerta.mx/Computo-Hardware/Servidores/Accesorios-para-Servidores/Cableado-Estructurado-para-Servidores/Organizadores-de-Cable/LinkPRO-Organizador-de-Cables-Horizontal-19-2U-Negro.html}{LinkPRO Organizador de Cables Horizontal 2U} & 3 & \$337 MXN & \$1,011 MXN\\
      \href{https://www.cyberpuerta.mx/Computo-Hardware/Redes/Access-Points/Access-Point-Open-Mesh-HPE-Networking-Instant-On-de-Banda-Dual-AP22-Inalambrico-1774-Mbit-s-1-Antena-Integrada-de-5-6dBi.html}{Access Point Aruba Instant On AP22 10.1W} & 9 & \$3,959 MXN & \$35,631 MXN\\
      \href{https://www.cyberpuerta.mx/Seguridad-y-Vigilancia/Camaras-y-Sistemas-de-Vigilancia/Camaras-de-Seguridad-IP/Ezviz-Camara-de-Seguridad-IP-Torreta-IR-para-Interiores-Exteriores-CS-H8C-3MP-POE-Alambrico-2304-x-1296-Pixeles-Dia-Noche.html}{Ezviz Cámara de Seguridad CS-H8C/3MP/POE 12W} & 12 & \$1,029 MXN & \$12,348 MXN\\
      \href{https://www.cyberpuerta.mx/Computo-Hardware/Cables/Bobinas/Bobinas-de-Cable-Ethernet/LinkedPRO-Bobina-de-Cable-Ethernet-Cat6-UTP-305-Metros-3.html?gad_source=1&gad_campaignid=22512493137&gbraid=0AAAAAD4nXn1OoM39AQ354wsCkyFu7PBmP&gclid=CjwKCAjw04HIBhB8EiwA8jGNbUPPV0I0JtFCylQ24yJ3d5XzZnZQm3WIaAlwiy4xL7B7PHQhPvgvgxoCArQQAvD_BwE}{Bobina Cable UTP Cat6 CMP 305m} & 5 & \$2,999 MXN & \$14,995 MXN\\
      \href{https://intercompras.com/p/cable-fibra-optica-multimodo-om3-hilos-osp-planta-externa-armada-con-179193?gad_source=1&gad_campaignid=22545716102&gbraid=0AAAAAD_EkoERYzRK4CQeGjV1BPMO_Lp5g&gclid=CjwKCAjw04HIBhB8EiwA8jGNbXL5nsc7dDSC95clrzIuRDVBuVleEboXCpaVKtXbs6hBuy3eo5cH7BoCsWMQAvD_BwE}{Fibra Óptica OM3 50/125 Armado 6 hilos (metro)} & 170 & \$87 MXN & \$14,790 MXN\\
      \href{https://intercompras.com/p/cable-fibra-optica-multimodo-om3-hilos-osp-planta-externa-dielectrico-179340}{Fibra Óptica OM3 50/125 General 6 hilos (metro)} & 20 & \$57 MXN & \$1,140 MXN\\
      \href{https://www.cyberpuerta.mx/Energia/Proteccion-Contra-Descargas/No-Break-UPS/No-Break-UPS/No-Break-Tripp-Lite-by-Eaton-SmartPro-Rack-Tower-UPS-300W-500VA-Entrada-120V.html}{No Break Tripp Lite 300W 500VA 1U} & 3 & \$5,219 MXN & \$15,657 MXN\\
      \href{https://www.cyberpuerta.mx/Computo-Hardware/Cables/Accesorios-para-Cables/Jacks-de-Red/Intellinet-Jack-Categoria-6-RJ-45-Blanco.html}{Keystone Intellinet RJ-45 Cat6} & 33 & \$49 MXN & \$1,617 MXN\\
      \href{https://www.cyberpuerta.mx/Computo-Hardware/Cables/Accesorios-para-Cables/Tapas-y-Cajas/Intellinet-Tapa-para-Caja-Faceplate-163286-Montaje-al-Ras-1-Salida-Blanco.html}{Faceplate Intellinet 163286} & 33 & \$28 MXN & \$924 MXN\\
      \href{https://www.cyberpuerta.mx/Celulares-y-Telefonia/Telefonia-VoIP/Telefonos-VoIP/Grandstream-Telefono-IP-GXP1625-2-Lineas-3-Teclas-Programables-Altavoz-Negro.html}{Grandstream Teléfono IP GXP1625} & 5 & \$769 MXN & \$3,845 MXN\\
      \href{https://www.cyberpuerta.mx/Energia/PDU-s/Saxxon-PDU-para-Rack-1U-SXPDU-12P16A-16A-125V-12-Salidas.html}{Saxxon PDU SXPDU-12P16A 1U 12 Salidas} & 3 & \$1,259 MXN & \$3,777 MXN\\
      \href{https://www.cyberpuerta.mx/Energia/Tierra-Fisica-y-Pararrayos/Tierra-Fisica/LinkedPRO-Barra-de-Distribucion-de-Tierra-Fisica-para-Rack-19-Cobre.html}{LinkedPRO Barra de Tierra Física} & 3 & \$1,099 MXN & \$3,297 MXN\\
      \href{https://www.cyberpuerta.mx/Energia/Tierra-Fisica-y-Pararrayos/Tierra-Fisica/LinkedPRO-Varilla-de-Tierra-LP-GROUND-60A-Cobre.html}{LinkedPRO Tierra Física LP-GROUND-60A} & 1 & \$5,719 MXN & \$5,719 MXN\\
      \href{https://www.cyberpuerta.mx/Computo-Hardware/Cables/Cables-de-Energia/Cables-de-Poder-Externo/Panduit-Cable-para-Aterrizar-Equipos-10AWG-para-Conexion-a-Tierra-0-61m-Verde-Amarillo.html}{Cable para Aterrizar Panduit 10AWG 0.61m} & 3 & \$719 MXN & \$2,157 MXN\\
      \href{https://www.amazon.com.mx/Mikrotik-Router-Ethernet-CCR2004-1G-12S-2XS/dp/B087X9QKGH/ref=asc_df_B087X9QKGH?mcid=487819bd58573ad391e06e226dd2caf7&tag=gledskshopmx-20&linkCode=df0&hvadid=709873468494&hvpos=&hvnetw=g&hvrand=14305191418972884946&hvpone=&hvptwo=&hvqmt=&hvdev=c&hvdvcmdl=&hvlocint=&hvlocphy=1010149&hvtargid=pla-934306539069&psc=1&language=es_MX&gad_source=4}{Router Mikrotik CCR2004-1G-12S+2XS} & 1 & \$11,609 MXN & \$11,609 MXN\\
      \textbf{Mano de obra} & 70\% costo de material & & \$107,650.90 MXN\\
\end{longtable}

\subsection{Identificación y Etiquetado}
\subsection{Identificación y etiquetado}

Conforme al estándar ANSI/TIA-606-B, se implementará un sistema de identificación para todos los elementos de infraestructura de telecomunicaciones del edificio, incluyendo dispositivos activos y componentes de cableado, con el objetivo de asegurar trazabilidad, mantenimiento eficiente y gestión del ciclo de vida de la red (Fries, 2011).

\paragraph*{Convención de identificadores.}
Cada componente de la red se etiqueta con un identificador que indica su ubicación, tipo de equipo y rol dentro del sistema. La estructura general es:

\[
\text{[Edificio]-[Piso][Sala]-[RackID]-[U] : [Puerto/Conector]}
\]

\noindent Esta convención permite un registro ordenado de los equipos y facilita la gestión de la infraestructura.

\paragraph*{Etiquetado de dispositivos activos.}
La convención se aplica a routers, switches, puntos de acceso, cámaras CCTV y teléfonos VoIP, garantizando que cada equipo cuente con un identificador único impreso.

\paragraph*{Etiquetado de cableado e infraestructura pasiva.}
Todos los enlaces de cableado horizontal y del backbone, así como los patch panels y racks, se identifican con la misma lógica para mantener coherencia entre el registro documental y la disposición física de los componentes (Fries, 2011).



\subsection{Cronograma}
\subsection{Cronograma de actividades}

\begin{figure}
	\centering
	\includegraphics[width=1\linewidth]{crono}
	\caption{}
	\label{fig:crono}
\end{figure}

\section{Implementación de la Red}
\subsection{Configuraciones de Equipos Activos}
\subsection{Configuraciones de equipos activos}

El archivo de Packet Tracer disponible en \href{https://drive.google.com/file/d/1mHfBjxOKNXVSxJJW_NH12IAQf1O9go8d/view?usp=sharing}{este enlace} contiene la implementación lógica de la red. Las principales configuraciones aplicadas son las siguientes:

\begin{itemize}
  \item Configuraciones básicas. Cada switch y router cuenta con un \texttt{hostname} único y contraseñas cifradas para acceso por consola, VTY y modo privilegiado.
  \item Configuración de VLAN. Se crean las VLAN 10 (ADMIN), 20 (ALUMNOS), 30 (PROFESORES), 40 (VOIP) y 50 (CÁMARAS) en el switch central. Los puertos de acceso se asignan a su VLAN correspondiente y los enlaces entre switches se configuran como \texttt{trunk}.
  \item Telefonía IP. Los teléfonos se ubican en la VLAN 40 con extensiones únicas y se verifica la comunicación entre ellos.
  \item Videovigilancia. Las cámaras CCTV emplean direcciones estáticas dentro del rango de la VLAN 50 para garantizar supervisión permanente.
  \item Direccionamiento IPv4. El plan VLSM define los gateways configurados en el switch multicapa del piso 1 para permitir el enrutamiento inter-VLAN.
\end{itemize}
\subsection{VLAN's}

% Buscar imágenes en la carpeta ../images (carpeta hermana al directorio configuracion)
\graphicspath{{../images/}}
		\textbf{Diagrama Lógico y Asignación de Red (VLSM)}
	\par \vspace{3mm}

	\begin{figure}[h!]
		\centering
		\includegraphics[width=0.9\textwidth]{diagrama-logico-red-vlans}
		
		\caption{\textit{Diagrama Lógico de Segmentación por VLANs}}
		\label{fig:diagrama-logico}
		
		\parbox{\textwidth}{
			\vspace{2mm}
			\small
			\textit{Nota.} La figura muestra el diagrama lógico de la red, codificado por colores para representar la segmentación de VLANs implementada. Esta separación lógica aísla el tráfico entre departamentos y servicios para mejorar el rendimiento y la seguridad.
		}
	\end{figure}
	
	\par \vspace{3mm}
	Para optimizar el uso del espacio de direcciones IP, se implementó un esquema de \textbf{Subnetting de Máscara de Longitud Variable (VLSM)} sobre el bloque de red base \texttt{148.220.0.0/21}. VLSM es una técnica que permite dividir un bloque de direcciones IP en subredes de diferentes tamaños, optimizando así la asignación de direcciones y minimizando el desperdicio (Singh \& Kaur, 2014).
	
	Como se observa en el diagrama (ver \textbf{Figura \ref{fig:diagrama-logico}}) y en el plan de direccionamiento anteriormente visto, las redes de usuarios (VLAN 20 Alumnos, VLAN 30 Profesores) se configuran con \textbf{DHCP} para una asignación automática de IPs. Por el contrario, la infraestructura (VLAN 10 Admin, VLAN 40 VoIP, VLAN 50 Cámaras) utiliza \textbf{asignación estática} para garantizar la seguridad. La primera IP utilizable de cada subred se reserva como \textbf{Gateway} y se configura en el Switch Core para permitir el enrutamiento entre las diferentes VLANs.
\subsection{Análisis de WiFi}
\input{configuracion/wifi.tex}
\subsection{Certificación de Cable}
\input{configuracion/certificacion-cable.tex}
\subsubsection{Reportes}
\input{configuracion/reportes.tex}
\section{Monitoreo y Análisis de la Red}
\section{Implementación de Seguridad}
\section{Administración de Fallas}
\section{Calidad en el Servicio}
\printbibliography
\end{document}
